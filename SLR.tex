\documentclass[12pt,a4paper]{article}

% Packages
\usepackage{times}
\usepackage{graphicx}
\usepackage{hyperref}
\usepackage{amsmath}
\usepackage{booktabs}
\usepackage{array}
\usepackage{longtable}
\usepackage{geometry}
\usepackage{microtype}

\usepackage[utf8]{inputenc}
\usepackage{newunicodechar}
\newunicodechar{≈}{\ensuremath{\approx}}

\geometry{margin=1in}

\title{Systematic Literature Review on the Use of AI Chatbots in Education and Learning Support}
\author{Chriscell Herojire Tumiwang - 220211060210 \\ 
	Jevi Ferdinan Monintja - 220211060226 \\ 
	Your Institution}
\date{}

\begin{document}
	\sloppy
	
	\maketitle
	
	\begin{abstract}
		Chatbot berbasis kecerdasan buatan (AI) kian diintegrasikan dalam ekosistem pendidikan sebagai tutor pribadi, asisten pembelajaran, dan mitra belajar. Namun, bukti tentang dampaknya terhadap hasil belajar dan keterlibatan, serta risiko etis dan integritas akademik, masih tersebar \cite{Labadze2023,Kuhail2023}.
		
		Metode: Kami melakukan tinjauan sistematis atas literatur 2018--2025, meliputi SLR generik tentang chatbot pendidikan, SLR khusus ChatGPT (PRISMA), dan meta-analisis efek chatbot pada pembelajaran. Sintesis naratif dilakukan untuk memetakan peran, konteks desain interaksi, outcome, dan tantangan, serta agenda riset ke depan.\cite{Labadze2023,Kuhail2023,Albadarin2024,su15042940}
		
		Hasil: Bukti konsisten menunjukkan chatbot menyediakan umpan balik instan, personalisasi konten dan kecepatan belajar, serta dukungan tugas dan konsep. Meta-analisis atas 32 studi (n$\approx$2201) menemukan efek sedang hingga tinggi pada hasil belajar agregat, dengan peningkatan bermakna pada prestasi, retensi, dan penalaran eksplisit. Namun, hasil negatif atau heterogen muncul pada kemampuan berpikir kritis, motivasi, dan keterlibatan. \cite{Labadze2023,Kuhail2023,su15042940,Saifullah2024,Baskara2023}
		
		Kesimpulan: Chatbot AI berpotensi memperkuat proses pembelajaran, tetapi manfaatnya bergantung pada desain instruksional, validasi keluaran, serta tata kelola etika. Penelitian mendatang perlu mengeksplorasi efek jangka panjang, moderator (usia, domain, durasi), serta desain scaffolding, kolaborasi manusia--AI, dan kebijakan integritas akademik.\cite{Labadze2023,Kuhail2023,Albadarin2024,su15042940}
	\end{abstract}
	
	\textbf{Keywords:} AI Chatbots, Education, Learning Support, SLR, PRISMA.
	
	\section{Introduction}
	Explain background, motivation, research gaps, and importance of AI chatbot research.
	
	\subsection{Research Questions}
	\begin{itemize}
		\item \textbf{RQ1}: What AI chatbot technologies are most commonly used in education?
		\item \textbf{RQ2}: In which contexts or domains are AI chatbots applied?
		\item \textbf{RQ3}: What benefits and limitations are reported?
		\item \textbf{RQ4}: What are the research trends over time?
	\end{itemize}
	
	\section{Background}
	Provide theoretical overview:
	\begin{itemize}
		\item Chatbots and Dialogue Systems
		\item NLP and LLM-based Chatbots
		\item Learning Support Systems
		\item Intelligent Tutoring Systems
	\end{itemize}
	
	\section{Methodology}
	This SLR follows PRISMA guidelines.
	
	\subsection{Search Strategy}
	Databases used:
	\begin{itemize}
		\item IEEE Xplore
		\item ACM Digital Library
		\item Scopus
		\item ScienceDirect
	\end{itemize}
	
	Example search string:
	\begin{verbatim}
		("chatbot" OR "AI chatbot" OR "conversational agent" OR "LLM") 
		AND ("education" OR "learning" OR "tutoring")
	\end{verbatim}
	
	\subsection{Inclusion Criteria}
	\begin{itemize}
		\item Focus on AI chatbots in education
		\item Published between 2018--2025
		\item Peer-reviewed articles
	\end{itemize}
	
	\subsection{Exclusion Criteria}
	\begin{itemize}
		\item Duplicate studies
		\item Non-English publications
		\item No access to full text
	\end{itemize}
	
	\subsection{Study Selection (PRISMA)}
	\begin{verbatim}
		Identified -> Screened -> Eligible -> Included
	\end{verbatim}
	
	\begin{figure}[h]
		\centering
		\fbox{\parbox{0.7\textwidth}{PRISMA Flow Diagram Placeholder}}
		\caption{PRISMA Flow Diagram}
	\end{figure}
	
	\section{Results}
	\subsection{Overview of Selected Studies}
	\begin{itemize}
		\item Total identified: XX
		\item Final included: XX
	\end{itemize}
	
	\section{Data Extraction Table}
	
	\begin{longtable}{|p{2.5cm}|p{1.5cm}|p{2cm}|p{3cm}|p{4cm}|}
		\hline
		\textbf{Author} & \textbf{Year} & \textbf{Domain} & \textbf{Technology Used} & \textbf{Key Findings} \\ \hline
		Author A & 2020 & Education & LLM Chatbot & Improved feedback. \\ \hline
		Author B & 2022 & Higher Ed & NLP Chatbot & Increased engagement. \\ \hline
	\end{longtable}
	
	\section{Discussion}
	Provide deeper analysis.
	
	\section{Conclusion}
	Summarize findings.
	
	\bibliographystyle{IEEEtran}
	\bibliography{references}
	
\end{document}
