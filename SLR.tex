\documentclass[12pt,a4paper]{article}

% Packages
\usepackage{times}
\usepackage{graphicx}
\usepackage{hyperref}
\usepackage{amsmath}
\usepackage{booktabs}
\usepackage{array}
\usepackage{longtable}
\usepackage{geometry}
\usepackage{microtype}

\usepackage[utf8]{inputenc}
\usepackage{newunicodechar}
\newunicodechar{≈}{\ensuremath{\approx}}

\geometry{margin=1in}

\title{Systematic Literature Review on the Use of AI Chatbots in Education and Learning Support}
\author{Chriscell Herojire Tumiwang - 220211060210 \\ 
	Jevi Ferdinan Monintja - 220211060226 \\ 
	Your Institution}
\date{}

\begin{document}
	\sloppy
	
	\maketitle
	
	\begin{abstract}
		Chatbot berbasis kecerdasan buatan (AI) kian diintegrasikan dalam ekosistem pendidikan sebagai tutor pribadi, asisten pembelajaran, dan mitra belajar. Namun, bukti tentang dampaknya terhadap hasil belajar dan keterlibatan, serta risiko etis dan integritas akademik, masih tersebar \cite{Labadze2023,Kuhail2023}.
		
		Metode: Kami melakukan tinjauan sistematis atas literatur 2018--2025, meliputi SLR generik tentang chatbot pendidikan, SLR khusus ChatGPT (PRISMA), dan meta-analisis efek chatbot pada pembelajaran. Sintesis naratif dilakukan untuk memetakan peran, konteks desain interaksi, outcome, dan tantangan, serta agenda riset ke depan.\cite{Labadze2023,Kuhail2023,Albadarin2024,su15042940}
		
		Hasil: Bukti konsisten menunjukkan chatbot menyediakan umpan balik instan, personalisasi konten dan kecepatan belajar, serta dukungan tugas dan konsep. Meta-analisis atas 32 studi (n$\approx$2201) menemukan efek sedang hingga tinggi pada hasil belajar agregat, dengan peningkatan bermakna pada prestasi, retensi, dan penalaran eksplisit. Namun, hasil negatif atau heterogen muncul pada kemampuan berpikir kritis, motivasi, dan keterlibatan. \cite{Labadze2023,Kuhail2023,su15042940,Saifullah2024,Baskara2023}
		
		Kesimpulan: Chatbot AI berpotensi memperkuat proses pembelajaran, tetapi manfaatnya bergantung pada desain instruksional, validasi keluaran, serta tata kelola etika. Penelitian mendatang perlu mengeksplorasi efek jangka panjang, moderator (usia, domain, durasi), serta desain scaffolding, kolaborasi manusia--AI, dan kebijakan integritas akademik.\cite{Labadze2023,Kuhail2023,Albadarin2024,su15042940}
	\end{abstract}
	
	\textbf{Keywords:} AI Chatbots, Education, Learning Support, SLR, PRISMA.
	
	\section{Introduction}
	Rencana kerja: menyusun Introduction berbahasa Indonesia berbasis SLR/PRISMA dengan sitasi ke referensi yang telah Anda cantumkan (menggunakan \cite{...} sesuai LaTeX), kemudian merumuskan RQ1–RQ4 yang operasional dan siap diekstraksi, serta menyediakan potongan LaTeX siap tempel.
	
	Introduction Kemajuan kecerdasan buatan (AI) dan pemrosesan bahasa alami (Natural Language Processing/NLP) telah mendorong adopsi chatbot sebagai tutor virtual, asisten belajar, dan mitra percakapan di berbagai jenjang pendidikan. Sejak 2018, terjadi pergeseran dari chatbot berbasis aturan dan alur percakapan statis menuju model berbasis pembelajaran mesin dan, lebih akhir, model bahasa besar (Large Language Models/LLM) yang mampu menyajikan umpan balik instan, penjelasan adaptif, serta personalisasi proses belajar \cite{Labadze2023,Kuhail2023}. Bukti-bukti terkini mengindikasikan bahwa chatbot dapat meningkatkan akses terhadap dukungan belajar, mempercepat siklus umpan balik, dan memfasilitasi elaborasi konsep; namun, dampaknya tidak selalu konsisten lintas konteks, topik, dan desain intervensi \cite{Labadze2023,Kuhail2023}.
	
	Di sisi lain, literatur menekankan tantangan etik dan integritas akademik, seperti risiko halusinasi keluaran, bias dan keadilan, privasi data, serta potensi penyalahgunaan pada tugas akademik. Sintesis dan meta-analisis terbaru melaporkan bahwa chatbot berasosiasi dengan peningkatan bermakna pada hasil belajar agregat—termasuk prestasi, retensi, dan penalaran eksplisit—namun menunjukkan heterogenitas pada motivasi, keterlibatan, dan kemampuan berpikir kritis \cite{su15042940,Saifullah2024,Baskara2023}. Kajian khusus mengenai adopsi sistem generatif seperti ChatGPT juga menegaskan perlunya desain instruksional yang hati-hati, verifikasi keluaran, dan kebijakan institusional untuk menjaga integritas dan keadilan akses \cite{Kuhail2023,Albadarin2024}.
	
	Menjawab kebutuhan pemetaan bukti yang komprehensif, studi ini menyajikan Tinjauan Sistematis (Systematic Literature Review/SLR) atas publikasi peer-reviewed periode 2018--2025 dengan mengikuti pedoman PRISMA. Pencarian terstruktur dilakukan pada basis data utama—IEEE Xplore, ACM Digital Library, Scopus, dan ScienceDirect—menggunakan kueri yang menggabungkan istilah chatbot/agen percakapan, AI/LLM, dan pendidikan/pembelajaran. Tujuan SLR ini adalah: (1) memetakan teknologi chatbot yang umum digunakan; (2) mengidentifikasi konteks dan domain penerapan; (3) mensintesis manfaat dan keterbatasan yang dilaporkan; serta (4) menelaah tren riset dari waktu ke waktu. Kontribusi yang ditargetkan meliputi peta lanskap terkini, rekomendasi desain dan tata kelola berbasis bukti, serta agenda riset ke depan yang memprioritaskan evaluasi jangka panjang, moderator efek (usia, domain, durasi), desain scaffolding dan kolaborasi manusia--AI, serta kebijakan integritas akademik \cite{Labadze2023,Kuhail2023,Albadarin2024,su15042940}.
	
	\subsection{Research Questions}
	\begin{itemize}
		\item \textbf{RQ1}: Teknologi chatbot AI apa yang paling umum digunakan dalam konteks pendidikan (misalnya berbasis aturan, retrieval-based, generative LLM), termasuk platform/layanan, arsitektur model, serta integrasi dengan Learning Management System (LMS) dan alat evaluasi?
		\item \textbf{RQ2}: Dalam konteks, jenjang, atau domain apa saja chatbot AI diterapkan (misalnya K-12, pendidikan tinggi, vokasional, STEM/non-STEM, pembelajaran formal/informal), dan bagaimana pengaturan pembelajarannya (tatap muka, daring, hibrida)?
		\item \textbf{RQ3}: Apa manfaat (misalnya prestasi belajar, retensi, kualitas umpan balik, efisiensi belajar) dan keterbatasan/risiko (misalnya bias, halusinasi, privasi, beban kognitif, integritas akademik) yang dilaporkan oleh studi-studi empiris?
		\item \textbf{RQ4}: Bagaimana tren riset 2018--2025 terkait volume publikasi, fokus metodologis (eksperimen, quasi-eksperimen, studi kualitatif, meta-analisis), tema topikal, serta pergeseran dari chatbot berbasis aturan menuju LLM generatif?
	\end{itemize}
	
	\section{Background}
	Provide theoretical overview:
	\begin{itemize}
		\item Chatbots and Dialogue Systems
		\item NLP and LLM-based Chatbots
		\item Learning Support Systems
		\item Intelligent Tutoring Systems
	\end{itemize}
	
	\section{Methodology}
	This SLR follows PRISMA guidelines.
	
	\subsection{Search Strategy}
	Databases used:
	\begin{itemize}
		\item IEEE Xplore
		\item ACM Digital Library
		\item Scopus
		\item ScienceDirect
	\end{itemize}
	
	Example search string:
	\begin{verbatim}
		("chatbot" OR "AI chatbot" OR "conversational agent" OR "LLM") 
		AND ("education" OR "learning" OR "tutoring")
	\end{verbatim}
	
	\subsection{Inclusion Criteria}
	\begin{itemize}
		\item Focus on AI chatbots in education
		\item Published between 2018--2025
		\item Peer-reviewed articles
	\end{itemize}
	
	\subsection{Exclusion Criteria}
	\begin{itemize}
		\item Duplicate studies
		\item Non-English publications
		\item No access to full text
	\end{itemize}
	
	\subsection{Study Selection (PRISMA)}
	\begin{verbatim}
		Identified -> Screened -> Eligible -> Included
	\end{verbatim}
	
	\begin{figure}[h]
		\centering
		\fbox{\parbox{0.7\textwidth}{PRISMA Flow Diagram Placeholder}}
		\caption{PRISMA Flow Diagram}
	\end{figure}
	
	\section{Results}
	\subsection{Overview of Selected Studies}
	\begin{itemize}
		\item Total identified: XX
		\item Final included: XX
	\end{itemize}
	
	\section{Data Extraction Table}
	
	\begin{longtable}{|p{2.5cm}|p{1.5cm}|p{2cm}|p{3cm}|p{4cm}|}
		\hline
		\textbf{Author} & \textbf{Year} & \textbf{Domain} & \textbf{Technology Used} & \textbf{Key Findings} \\ \hline
		Author A & 2020 & Education & LLM Chatbot & Improved feedback. \\ \hline
		Author B & 2022 & Higher Ed & NLP Chatbot & Increased engagement. \\ \hline
	\end{longtable}
	
	\section{Discussion}
	Provide deeper analysis.
	
	\section{Conclusion}
	Summarize findings.
	
	\bibliographystyle{IEEEtran}
	\bibliography{references}
	
\end{document}
